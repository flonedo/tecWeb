\documentclass[openany,10pt,a4paper]{article}
\usepackage[T1]{fontenc} % codifica dei font
\usepackage[utf8]{inputenc} % lettere accentate da tastiera
\usepackage[italian]{babel} % lingua del documento
\usepackage{url} % per scrivere gli indirizzi Internet
\usepackage{graphicx}
\usepackage{subfiles}



\title{ScambioLibriVI}
\author{Luca Allegro 1120573 \and Marco Giollo 1122414 \and Francesca Lonedo 1125508}
\date{\today}


\begin{document}
	\makeatletter
	\begin{titlepage}
		\begin{center}
			

			{\Huge \bfseries  \@title }\\[5em]
			\vfill
			{\Large Progetto di Tecnologie Web} \\[2em]
			\begin{tabular}{ p{5cm} | p{5cm} }
				Anno Accademico & 2017-2018 \\
				\\
				Autori & Luca Allegro \newline Marco Giollo \newline Francesca Lonedo \\
				\\
				Professori & Lamberto Ballan \newline Matteo Ciman \newline Ombretta Gaggi  \\
				\\
				Sito & \url{http://tecweb2016.studenti.math.unipd.it/flonedo/} \\
				\\
				E-mail & marco.giollo@studenti.unipd.it \newline scambiolibrivi@gmail.com \\
				\\
			\end{tabular}\\[3 em]
			\vfill
			{\large \@date}
			\vfill
			\includegraphics[height=2 cm]{header_dipartimento_matematica.png}\\[3em]
		\end{center}
	\end{titlepage}


	\pagenumbering{gobble}
	
	


	\newpage

	\pagenumbering{roman}
	\tableofcontents
	\newpage
	\pagenumbering{arabic}

\section{Introduzione}
	\subsection{Cos'è ScambioLibriVI}
		\textit{ScambioLibriVI} è una piattaforma web pensata per la compravendita di libri usati tra privati. \\
		Poiché le elevate spese di spedizione rendono spesso inconveniente l'acquisto dell'usato rispetto al nuovo, con \textit{ScambioLibriVI} si è deciso di abbandonare il classico concetto di e-Commerce: il sito vuole infatti essere una bacheca degli annunci per mettere in contatto venditori e acquirenti, che dovranno accordarsi autonomamente sulle modalità di scambio. Sebbene gli utenti siano in totale libertà, lo spirito dell'applicazione è quello di favorire il passaggio "a mano" dei prodotti in quanto sicuro, conveniente ed ecologico.
	\subsection{L'idea di base}
	La compravendita di libri usati è attualmente un'attività difficile; sebbene esistano piattaforme che offrano tale servizio, questo è spesso limitato. L'offerta attualmente disponibile include:
	\begin{itemize}
		\item piattaforme di e-commerce generiche;
		\item negozi online che trattengono una percentuale del prezzo;
		\item piccoli forum specializzati di scarsa diffusione.
	\end{itemize}
	E' in questo contesto che si inserisce \textit{ScambioLibriVI}, una piattaforma gratuita che si offre di mettere in contatto gli utenti interessati a scambiare libri.
\section{Progettazione}
	\subsection{Requisiti}
		La piattaforma prevede due categorie di utenti, che hanno accesso a differenti funzionalità.
		\subsubsection{Visitatore}
		Un visitatore del sito ha accesso alle seguenti funzionalità
				\begin{itemize}
					\item creazione di un nuovo account:
						\begin{itemize}
							\item inserimento username;
							\item inserimento password;
							\item inserimento nome;
							\item inserimento cognome;
							\item inserimento città;
							\item inserimento provincia;
							\item inserimento indirizzo e-mail;
							\item inserimento numero di telefono facoltativo.
						\end{itemize}
					\item consultazione del catalogo;
					\item visualizzazione dei dettagli relativi a un libro:
						\begin{itemize}
							\item ISBN;
							\item titolo;
							\item autore;
							\item stato del libro;
							\item casa editrice;
							\item anno di pubblicazione;
							\item zona di ritiro del libro;
							\item generi di appartenenza;
							\item eventuale immagine di copertina;
							\item informazioni sul venditore:
								\begin{itemize}
									\item username;
									\item nome;
									\item cognome;
									\item città;
									\item provincia;
									\item numero di telefono, se presente;
									\item contatto diretto via e-mail;
								\end{itemize}
							
						\end{itemize}
				\end{itemize}		
		\subsubsection{Utente registrato}
		Una volta eseguito l'accesso al sito inserendo \textbf{username} e \textbf{password}, un utente ha accesso al proprio \textbf{profilo riservato}. Il pannello del profilo riservato mette a disposizione le 
		seguenti funzionalità:
				\begin{itemize}
					\item modifica dei dati associati al profilo:
						\begin{itemize}
							\item cambio password;
							\item modifica città;
							\item modifica provincia;
							\item modifica indirizzo e-mail;
							\item modifica numero di telefono;
						\end{itemize}
					\item inserimento di un nuovo libro da vendere:
						\begin{itemize}
							\item inserimento codice ISBN;
							\item inserimento titolo;
							\item inserimento autore;
							\item inserimento casa editrice;
							\item inserimento anno di pubblicazione;
							\item inserimento prezzo;
							\item selezione dei generi di appartenenza (massimo 3);
							\item selezione dello stato di usura;
							\item inserimento di note aggiuntive;
							\item caricamento di un'immagine;
						\end{itemize}
					\item rimozione di un libro tra quelli inseriti;
					\item rimozione del profilo personale.
				\end{itemize}
		\subsubsection{Altri requisiti di sistema}	
		Oltre alle funzionalità specifiche sopra descritte il sito deve garantire:
			\begin{itemize}
				\item \textbf{accessibilità:}
					\begin{itemize}
						\item dispositivi mobile;
						\item utenti diversamente abili;
					\end{itemize}
				\item \textbf{usabilità:}
					\begin{itemize}
						\item utilizzo intuitivo;
						\item facile accesso alle informazioni;
					\end{itemize}
			\end{itemize}
			
	\subsection{Scelte implementative}
		\subsubsection{Database}
			Per la gestione dei dati degli utenti e dei libri inseriti, \textit{ScambioLibriVi} si avvale di un 
			database relazione \textit{MySQL}.
			E' stato scelto di utilizzare un database relazionale, in particolare \textit{MySQL}, per la facilità di gestione dello stesso rispetto ad altre tecnologie; un database \textit{MySQL}, infatti, risiede in un unico server, evitando così problematiche relative alla consistenza dei dati. \\ La scelta di un database centralizzato pone ovviamente dei limiti legati alla scalabilità del sistema e alla sua affidabilità. Poiché il carico di utenti previsto, almeno nelle prime fasi di rilascio del sistema, è ben lontano dalla soglia critica, non sono previste problematiche legate alla troppa affluenza nell'immediato futuro; poiché il servizio non tratta transazioni monetarie o funzionalità che richiedano risposte real-time, la sua vulnerabilità a guasti o attacchi che ne impediscano il funzionamento momentaneo è stata considerata un rischio trascurabile.
		\subsubsection{Pagine web}
			Per la realizzazione di  \textit{ScambioLibriVi} sono stati utilizzati \textit{XHTML}, \textit{CSS3}, \textit{PHP} e \textit{JavaScript}. \\
			Le pagine vengono realizzate dinamicamente, in modo da poter riutilizzare tutte le porzioni di codice condivise come \textit{header}, \textit{footer}, \textit{login}, \textit{menu} ecc; questo ha permesso di ridurre il peso complessivo del sito e facilità notevolmente le attività di manutenzione. Ad essere creato dinamicamente mediante l'utilizzo di codice \textit{PHP} è anche il contenuto richiesto dall'utente, tramite chiamate al database.
			\\ La grafica del sito è gestita interamente in \textit{CSS3}; sono resi disponibili tre fogli di stile che forniscono il supporto alla consultazione tramite dispositivi desktop e mobile; un ulteriore foglio di stile è dedicato alla stampa. 
			\\ L'utilizzo di codice \textit{JavaScript} è ridotto al minimo per garantire il più alto grado di accessibilità raggiungibile. E' stata prestata particolare attenzione a permettere la navigazione del sito anche da dispositivi che non supportino \textit{JavaScript}; in particolare il sito è accessibile anche mediante l'utilizzo di \textit{screen readers}, quali il lettore integrato di \textit{Microsoft Edge}.
			\\ Grazie a questi accorgimenti vi è una totale divisione tra il contenuto (XHTML), la sua presentazione (CSS3) e il suo comportamento (PHP).
			
		\subsubsection{Accessibilità}
		E' stata curata in dettaglio l'accessibilità del sito da parte di tutte le categorie di utenti, tra cui i diversamente abili. In particolare sono stati presi i seguenti accorgimenti:
			\begin{itemize}
				\item inserimento di ausili alla navigazione;
				\item riduzione al minimo dell'utilizzo di \textit{JavaScript} e trasformazione elegante e completa di tutte le funzionalità qualora non disponibile; per come è stato utilizzato, \textit{JavaScript} non influisce negativamente sulla performance degli \textit{screen readers};
				\item dichiarazione di span di lingua per facilitare la lettura del contenuto tramite \textit{screen reader}; per ovvi motivi, non è possibile garantire la corretta lettura dei contenuti inseriti dagli utenti, che verranno sempre letti con pronuncia italiana;
				\item è stato completamente evitato l'utilizzo di tabelle per la strutturazione del contenuto; sono stati invece utilizzati elenchi puntati, meglio gestiti dagli \textit{screen reader};
				\item il contenuto è totalmente separato dalla sua presentazione, gestita tramite \textit{CSS}; sono forniti due fogli di stile per adattare il sito alla visualizzazione tramite dispositivi desktop e mobile ed è definito un ulteriore foglio per la stampa; il sito è pienamente utilizzabile anche senza presentazione \textit{CSS};
				\item scelta di una palette di colori a contrasto elevato per garantire l'accessibilità ad utenti affetti da daltonismo.
			\end{itemize}
		\textit{ScambioLibriVi} è validato XHTML strict secondo lo standard del W3C; anche lo stile CSS è validato W3C.
		\\
		Riportiamo in seguito i risultati dei test di simulazione del daltonismo, eseguiti tramite il tool fornito da \url{http://www.vischeck.com}: \\
		\\
		\includegraphics[height=4 cm]{cb1.jpg}\\[3em]
		\includegraphics[height=4 cm]{cb2.jpg}\\[3em]
		\includegraphics[height=4 cm]{cb2.jpg}\\[3em]
			
			
		\subsubsection{Usabilità}
		Il sito è stato reso semplice ed intuitivo in modo da permetterne l'utilizzo alle più varie categorie di utenti appassionate di lettura e questo ha ridotto il numero di funzionalità che è stato deciso di rendere disponibile. \\ 
		Lo scopo di \textit{ScambioLibriVi} è quello di invogliare gli utenti a scoprire qualcosa di nuovo e aiutare appassionati a scambiare libri tra di loro, e non vuole in alcun modo sostituire una tradizionale libreria o e-commerce.
		Per la natura del servizio, quindi, è stato scelto consapevolmente di non fornire uno strumento di ricerca specifica: così come in una biblioteca il visitatore guarda cosa è disponibile o meno aggirandosi tra gli scaffali dei vari reparti, anche in \textit{ScambioLibriVi} l'utente sfoglia il catalogo a seconda del genere di interesse. 
			A tal proposito è stata inoltre messa a disposizione nella home del sito una bacheca delle novità.
		Date le dimensioni ridotte del sito e le motivazioni sopra esposte, riteniamo che l'assenza di opzioni di ricerca avanzate non sia al momento una limitazione; anche con il solo filtro per genere, infatti, in pochi click è comunque possibile scoprire se un eventuale particolare testo cercato sia presente o meno nel catalogo.
	
	\section{Team di sviluppo}
	\textit{ScambioLibriVi} è stato interamente progettato ed implementato da Luca Allegro, Marco Giollo e Francesca Lonedo. E' segnato in \textbf{grassetto} il membro del gruppo che ha fornito il contributo maggiore all'attività.
	\begin{itemize}
		\item Progettazione: Luca Allegro, Marco Giollo, Francesca Lonedo;
		\item Database: 
			\begin{itemize}
				\item creazione: \textbf{Luca Allegro}, Marco Giollo, Francesca Lonedo;
				\item popolamento: \textbf{Francesca Lonedo};
			\end{itemize}
		\item XHTML:
			\begin{itemize}
				\item testi: Luca Allegro, Marco Giollo, \textbf{Francesca Lonedo};
				\item struttura: Luca Allegro, \textbf{Marco Giollo}, Francesca Lonedo;
				\item divisione contenuto e comportamento: Luca Allegro, \textbf{Marco Giollo}, Francesca Lonedo;
			\end{itemize}
		\item PHP:
			\begin{itemize}
				\item area utente: \textbf{Luca Allegro}, Marco Giollo;
				\item scheda libri: \textbf{Marco Giollo}, Francesca Lonedo;
				\item catalogo: \textbf{Francesca Lonedo};
				\item funzioni e chiamate al database: Luca Allegro, \textbf{Marco Giollo}, Francesca Lonedo;
			\end{itemize}
		\item JavaScript: Luca Allegro, \textbf{Marco Giollo}, Francesca Lonedo;
		\item Fogli di stile CSS:
			\begin{itemize}
				\item desktop: Luca Allegro, \textbf{Marco Giollo}, Francesca Lonedo;
				\item mobile: \textbf{Francesca Lonedo};
				\item stampa: \textbf{Luca Allegro};
			\end{itemize}
		
		\item Relazione: \textbf{Francesca Lonedo};
		\item Test: Luca Allegro, Marco Giollo, Francesca Lonedo.
	\end{itemize}

\section{Caricamento}
	Il sito web è disponibile all'indirizzo \url{http://tecweb2016.studenti.math.unipd.it/flonedo/}. Le credenziali di accesso all'area riservata sono username: user; password: user. E' comunque possibile aggiungere nuovi utenti a discrezione.
\end{document}
